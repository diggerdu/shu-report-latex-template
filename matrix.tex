\documentclass{zjureport}
% =============================================
% Part 1 Edit the info
% =============================================

\dmajor{计算机科学}
\dname{颜乐春}
% 假装封装了一下前面的变量定义部分

\newcommand{\stuid}{15124542}
\newcommand{\newdate}{2017-10-09}
\newcommand{\loc}{704}

\newcommand{\course}{算法设计於分析}
\newcommand{\tutor}{神人}
\newcommand{\grades}{59}
\newcommand{\newtitle}{棋盘覆盖问题}
\newcommand{\exptype}{Null}
\newcommand{\group}{None}

% 下划线的一个定义

  
% =============================================
% Part 1 Main document
% =============================================
\begin{document}
\thispagestyle{empty}

\makeCover
%封装了一下这部分
%还把下划线弄了一下

% =============================================
% Part 2 Main document
% =============================================

\section{问题描述於实验目的}
给定$n$个矩阵$A_{1},A_{2},\cdots,A_n$,其中,$A_i$与$A_{j+1}$是可乘的,$i=1,2,\cdots,n-l$。
你的任务是要确定矩阵连乘的运算次序,使计算这$n$个矩阵的连乘积$A_{1}A_2\cdots A_n$时总的元素乘法次数达到最少。
例如:3个矩阵$A_1,A_2,A_3$阶分别为$10 \times 100、100 \times 5、5 \times 50$,计算连乘积$A_1A_2A_3$时按$(A_1 A_2 A_3$所需的元素乘法次数达到最少,为$7500$次。
\subsection{输入}
测试数据有若干组,每组测试数据有2行。\par
每组测试数据的第$1$行是一个整数$n$,$(0<n<20)$,第$2$行是$n+1$个正整数$p_0、p_1、p_2、\cdots、p_n$,这些整数不超过100,相邻两个整数之间空一格,他们表示$n$个矩阵$A_1,A_2,\cdots,A_n$,的阶$p_{i-1}, p_{i},i=1,2,\cdots,n$.\par
输入直到文件结束。\par
\subsection{输出}
对输入中的每组测试数据,输出2行。先在一行上输出“Case Num”,其中“Num”是测试数据的组号(从1开始),再在第2行上输出计算这$n$个矩阵的连乘积$A_1 A_2 \cdots A_n$时最少的总的元素乘法次数,再空一格,接着在同一行上输出矩阵连乘的添括号形式。
注意:最外层括号应去掉。
\subsection{输入样例}
\begin{lstlisting}
3
10 100 5 50
4
50 10 40 30 5
\end{lstlisting}

\subsection{输出样列}
\begin{lstlisting}
Case 1
7500 (A1A2)A3
Case 2
10500 A1(A2(A3A4))
\end{lstlisting}

\section{实验环境}
Ubuntu 17.04 + gcc 6.3

\section{实验内容和步骤}
	\subsection{设计思路}
		计算A[i:j]的最优次序所包含的计算矩阵子琏A[i:k]和A[k+1:j]的次序也是最优的,也即是说这个问题具有最优子结构性质,可以用动态规划解决。下面是状态转移方程:\par
		\begin{equation}
			m[i,j]= \min_{i \leqslant k < j} {m[i,k] + m[k+1,j] + p_{i-1}p_i}
		\end{equation}

\section{实现程序}
	\begin{clause}
		\lstinputlisting[language=C++]{code/cheese.cpp}
	\end{clause}


\end{document}
