\documentclass{zjureport}
% =============================================
% Part 1 Edit the info
% =============================================

\dmajor{计算机科学}
\dname{颜乐春}
% 假装封装了一下前面的变量定义部分

\newcommand{\stuid}{15124542}
\newcommand{\newdate}{2017-10-09}
\newcommand{\loc}{704}

\newcommand{\course}{算法设计於分析}
\newcommand{\tutor}{神人}
\newcommand{\grades}{59}
\newcommand{\newtitle}{棋盘覆盖问题}
\newcommand{\exptype}{Null}
\newcommand{\group}{None}

% 下划线的一个定义

  
% =============================================
% Part 1 Main document
% =============================================
\begin{document}
\thispagestyle{empty}

\makeCover
%封装了一下这部分
%还把下划线弄了一下

% =============================================
% Part 2 Main document
% =============================================

\section{问题描述於实验目的}
给定$n$个字母(或字)在文档中出现的频率序列$X=<x_1,x_2,…,x_n>$,求出这n个字母的Huffman编码。为方便起见,以下将频率用字母出现的次数(或称权值)$w_1,w_2,\cdots,w_n$代替。\par
\subsection{输入}
输入文件中的开始行上有一个整数$T$,$(0<T\leqslant20)$,表示有$T$组测试数据。\par
接下来是$T$行测试数据的描述,每组测试数据有$2$行。测试数据的第$1$行上是一个正整数$n$,$(n<50)$,表示序列的长度。第$2$行是$n$个字母出现的权值序列$w_1,w_2,\cdots,w_n$,它们均为正整数,相邻的两个整数之间用空格隔开。\par
输入直到文件结束。\par

\subsection{输出}
对输入中的每组有$n$个权值的$ica$数据,应输出$n+1$行:先在一行上输出“Case Num”,其中“Num”是测试数据的组号(从1开始);接下来输出$n$行,其第$1$行到第$n$行上依次输出第i个字母出现的次数和相应的Huffman编码,格式如下:\\
$w_i$ Huffman编码。\\
每组测试数据对应的输出最后结束时加一个空行,以便区分。\\
为保证Huffman编码的唯一性,在构造Huffman树的过程中,我们约定:\\
\begin{itemize}
\item 左儿子标记为0,右儿子标记为1;
\item 左儿子的权值>=右儿子的权值;
\item 相同权值w的两个字母x、y,先输入权值的字母x的Huffman编码长度不超过后输入权值的字母y的Huffman编码长度。
\item 合并两个节点后新的权值应从右到左搜索、插入到相应的位置。
\end{itemize}


\subsection{输入样例}
\begin{lstlisting}
2
6
9 8 3 4 1 2
8
60 20 5 5 3 3 3 1
\end{lstlisting}

\subsection{输出样列}
\begin{lstlisting}
Case 1
9 00
8 01
3 100
4 11
1 1011
2 1010

Case 2
60 0
20 10
5 1101
5 1110
3 11000
3 11001
3 11110
1 11111
\end{lstlisting}

\section{实验环境}
Ubuntu 17.04 + gcc 6.3

\section{实验内容和步骤}
	\subsection{设计思路}
	这个算法是通过为每个顶点$v$保留目前为止所找到的从s到v的最短路径来工作的。初始时,原点$s$的路径权重被赋为$0 (d[s] = 0)$。若对于顶点$s$ 存在能直接到达的边$(s,m)$,则把$d[m]$设为$w(s, m)$,同时把所有其他($s$不能直接到达的)顶点的路径长度设为无穷大,即表示我们不知道任何通向这些顶点的路径(对于所有顶点的集合 $V $中的任意顶点 $v$, 若 $v$ 不为 $s$ 和上述 $m$ 之一, $d[v] = \inf$)。当算法结束时,$d[v]$ 中存储的便是从 $s$ 到 $v$ 的最短路径,或者如果路径不存在的话是无穷大。
	\newpage
	\subsection{算法描述}
		\begin{algorithm}[h!]
		\caption{Dijkstra's Algorithm} 
		\begin{algorithmic}[1]
			\Function {Dijkstra}{Graph, source}
				\State create vertex set Q
				\For{vertex v in Graph}
					\State $dist[v] \gets INFINITY$
					\State $prev[v] \gets UNDEFINED$
					\State $add v to Q$
				\EndFor

				$dist[source] \gets 0$
				\While{Q is not empty}
					\State $u \gets vertex in Q with min dist[u]$
					\State $remove u from Q$
					\For{neighbor v of u}
						$alt \gets dist[u] + length(u,v)$
						\If{alt < dist[v]}
							$dist[v] \gets alt$
							$prev[v] \gets u$
						\EndIf
					\EndFor
				\EndWhile
				\Return dist prev
			\EndFunction
		\end{algorithmic}
		\end{algorithm}
\section{实现程序}
	\begin{clause}
		\lstinputlisting[language=C++]{code/dijkstra.cpp}
	\end{clause}


\end{document}
