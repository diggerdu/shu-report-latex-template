\documentclass{zjureport}
% =============================================
% Part 1 Edit the info
% =============================================

\dmajor{计算机科学}
\dname{颜乐春}
% 假装封装了一下前面的变量定义部分

\newcommand{\stuid}{15124542}
\newcommand{\newdate}{2017-10-09}
\newcommand{\loc}{704}

\newcommand{\course}{算法设计於分析}
\newcommand{\tutor}{神人}
\newcommand{\grades}{59}
\newcommand{\newtitle}{棋盘覆盖问题}
\newcommand{\exptype}{Null}
\newcommand{\group}{None}

% 下划线的一个定义

  
% =============================================
% Part 1 Main document
% =============================================
\begin{document}
\thispagestyle{empty}

\makeCover
%封装了一下这部分
%还把下划线弄了一下

% =============================================
% Part 2 Main document
% =============================================

\section{问题描述於实验目的}
序列$Z=<B,C,D,B>$是序列$X=<A,B,C,B,D,A,B>$的子序列,相应的递增下标序列为$<2,3,5,7>$。\par
一般地,给定一个序列$X=<x_1,x_2,\cdots,x_m>$,则另一个序列$Z=<z_1,z_2,\cdots,z_k>$是X的子序列,是指存在一个严格递增的下标序列$〈i1,i2,…,ik〉$使得对于所有$j=1,2,…,k$使$Z$中第$j$个元素$z_j$与$X$中第$i_j$个元素相同。
给定$2$个序列$X$和$Y$,当另一序列$Z$既是$X$的子序列又是$Y$的子序列时,称$Z$是序列$X$和$Y$的公共子序列。\par
你的任务是:给定$2$个序列$X、Y$,求$X$和$Y$的最长公共子序列$Z$。\par


\subsection{输入}
输入文件中的第1行是一个正整数$T,(0 < T \leqslant 10)$,表示有$T$组测试数据。接下来是每组测试数据的描述,每组测试数据有$3$行。\par
测试数据的第$1$行有$2$个正整数$m、n$,中间用一个空格隔开,$(0<m,n<50)$;第$2、3$行是长度分别为$m、n$的$2$个序列$X$和$Y$,每个序列的元素间用一个空格隔开。序列中每个元素由字母、数字等构成。\par
输入直到文件结束。\par

\subsection{输出}
对输入中的每组测试数据,输出2行。先在一行上输出“Case Num”,其中“Num”是测试数据的组号(从1开始),再在第2行上输出这2个序列$X、Y$的最长公共子序列$Z$的长度及子序列$Z$(至少一个)。\par

\subsection{输入样例}
\begin{lstlisting}
2
7 6
A B C B D A B
B D C A B A
8 9
b a a b a b a b
a b a b b a b b a
\end{lstlisting}

\subsection{输出样列}
\begin{lstlisting}
Case 1
4 LCS(X,Y):B C B A
Case 2
6 LCS(X,Y):a b a b a b
\end{lstlisting}

\section{实验环境}
Ubuntu 17.04 + gcc 6.3

\section{实验内容和步骤}
	\subsection{设计思路}
	最长公共子序列问题有最优子结构性质。\par
	记:\\
	$X_i=﹤x_1,\cdots,x_i﹥$即$X$序列的前$i$个字符 ($1\leqslant i \leqslant m$)(前缀)\\
	$Y_j=﹤y_1, \cdots,y_j﹥$即$Y$序列的前$j$个字符 ($1 \leqslant j \leqslant n$)(前缀)\\
	假定$Z=﹤z_1,\cdots,z_k﹥ \in LCS(X , Y)$。\\
	若$x_m=y_n$(最后一个字符相同),则不难用反证法证明:该字符必是$X$与$Y$的任一最长公共子序列$Z$(设长度为$k$)的最后一个字符,即有$z_k = x_m = y_n$ 且显然有$Z+k-1 \in LCS(X_{m-1} , Y_{n-1})$即$Z$的前缀$Z_{k-1}$是$X_{m-1}$与$Y_{n-1}$的最长公共子序列。此时,问题化归成求$X_{m-1}$与$Y_{n-1}$的$LCS(LCS(X , Y)$的长度等于$LCS(X_{m-1} , Y_{n-1})$的长度加$1$)。\par
	若$x_m \neq y_n$,则亦不难用反证法证明:要么$Z \in LCS(X_{m-1}, Y)$,要么$Z \in LCS(X , Y_{n-1})$。由于$z_k \neq x_m$与$z_k \neq y_n$其中至少有一个必成立,若$z_k \neq x_m$则有$Z \in LCS(X_{m-1} , Y)$,类似的,若$zk \neq yn$ 则有$Z \in LCS(X , Y_{n-1})$。此时,问题化归成求$X_{m-1}$与$Y$的$LCS$及$X$与$Y_{n-1}的$LCS$。$LCS(X , Y)$的长度为:$max{LCS(X_{m-1} , Y)$的长度, $LCS(X , Y_{n-1})$的长度}$。\par
	由于上述当$x_m \neq y_n$的情况中,求$LCS(X_{m-1} , Y)$的长度与$LCS(X , Yn-1)$的长度,这两个问题不是相互独立的:两者都需要求$LCS(Xm-1,Yn-1)$的长度。另外两个序列的$LCS$中包含了两个序列的前缀的$LCS$,故问题具有最优子结构性质考虑用动态规划法。


\section{实现程序}
	\begin{clause}
		\lstinputlisting[language=C++]{code/lcs.cpp}
	\end{clause}


\end{document}
